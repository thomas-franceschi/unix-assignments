\documentclass{article}

\title{Reading 08: Document Tools}
\date{14 March, 2016}
\author{Thomas Franceschi}

\usepackage{graphicx}
\usepackage{hyperref}
\usepackage[margin=1in]{geometry}

\begin{document}

\maketitle

%------------------------------------------------------

\section*{Overview}

For this expriment, I created {\bf three} scripts:

\begin{enumerate}

\item{\tt rolldice.sh}: This script simulates rolling a dice.

\item{\tt experiment.sh}: This script uses rolldice.sh to perform an experiment and then collect that data into results.dat

\item{\tt histogram.plt}: This script uses gnuplot to create a graph of the data in results.dat.

\end{enumerate}

\section*{Rolling Dice}

First, I created a script called roll dice.sh that uses the shuf command to simulate rolling a dice and set up flags
to modify the number of sides and the number of rolls to be done. Each "roll" randomly shuffles a list of all possible 
numbers on the dice then I use the head function to record only the first member of the list as the "winner".

\begin{verbatim}
$ ./roll_dice.sh -h
usage: roll_dice.sh [-r ROLLS -s sides]

-r ROLLS    Number of rolls of die (default: 10)
-s SIDES    Number of sides on die (default: 6)

\end{verbatim}

\section*{Experiment}

Second, I created a script called experiment.sh that uses the roll dice.sh and the -r flag with 1000 as the modifier to simulate 
1000 rolls of the dice. I then used {\bf awk} to sum the occurnces of each number that was rolled and output it to a file called results.dat








\section*{Results}

Table 1 contains the results of my experiment of rolling a dice 1000 times.

\begin{table}[h!]
    \centering
    \begin{tabular}{r||c}
    Side	& Counts\\
    \hline
    1	& 151\\
    2	& 174\\
    3	& 172\\
    4	& 178\\
    5	& 156\\
    6	& 169\\
    \end{tabular}
    \caption{Example Tables}
    \label{tbl:example}
\end{table}

\begin{figure}[h!]
\centering
\includegraphics[width=\textwidth]{results.png}
\caption{Graph}
\label{fig:graph}
\end{figure}

\end{document}