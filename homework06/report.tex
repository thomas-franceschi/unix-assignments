\documentclass{article}

\title{Gender and Ethnicity in Computer Science at Notre Dame}
\date{18 March, 2016}
\author{Thomas Franceschi \\ tfrances@nd.edu}

\usepackage{float}
\usepackage{graphicx}
\usepackage{hyperref}
\usepackage[margin=.8in]{geometry}

\begin{document}

\maketitle

%------------------------------------------------------

\section*{Overview}

For this homework, I had to parse the data in a csv file into data files and then readable graphs to understand the data 
being pesented in them. The data being observed was the gender and racial composition of students 
studying computer science at Notre Dame. I wrote two shell scripts to parse the gender and ethnicity data 
respectively that used i/o redirection and piping to read the csv file and turn it into meaningful, human 
readable data with counts of students of eachgender and ethnicity according to graduation year. I then wrote to 
gnuplot scripts to convert the .dat files into plots which make the data much more easy to visualize and understand.

\section*{Methodology}

I wrote two shell scripts (gender.sh and ethnicity.sh) which each parsed the 
file and tallied the number of people of each gender and ethnicity according to graduation year. I 
accomplished this by echoing a header of "Year Men Women" seperated by tabs and directing it into an 
output file named gender.dat, and the same for ethnicity.dat but with the different ethnicities as the 
headers. Once those new files are created I have a while loop that runs as long as the year variable is 
less than 2019 (it is initialized to 2013). I then cat the file to be parsed, pipe it into a cut command 
that cuts the gender (or ethnicity) column of the corresponding year, then use grep to filter for only 
the letters to be counted (MF/COSBNTU). After that I use awk to count the instances of each lettter for 
each graduating year and print the tally next to the corresponding year and under the corresponding group 
header and append it to the .dat file that was initially created. The loop then continues until it has 
gone through all possible years. One interesting problem I found was being able to genereate a .dat file 
then running the script again and not just adding to the existing file but creating a new one. I solved this 
by generating the header before the while loop so it overwrites any previous .dat file of the same name, since 
all of the data being parsed is simply appended to an existing file.\\ \\
To create the graphs, I use gnuplot and set a static range, set the style to a histogram cluster, set the 
terminal png output size and destination of the output (gender.png/ethnicity.png), along with some other 
style settings, and then I plot the corresponding columns (2:3 for gender and 2:8 for ethnicity) using the 
first line of each entry column as the title. One problem I found was when it was generating the key it would split 
African American into two groups and list "African" where "African American" should be and "American" where 
"Native American" should be. I solved this by simply replacing the space with an underscore so it is read as one continuous 
word. I am sure there is a less hackey way to solve this issue that retains the space, but that was the easiest 
solution I could come up with.

\section*{Analysis}

In this section you can see tables and graphs visualizing the gender and ethnic make up of the students in the Computer 
Science major at Notre Dame. As it is plainly obvious, Caucasian male is the largest group of students with both the 
numbers of male and white students growing year over year very consistently. The number of women is consistently 
growing as well, but not at a pace where it seems they can catch up to men any time soon. The number of non-white 
students appears to not be increasing nearly as much as the white students and seems mostly stagnant.

\begin{table}[H]
\centering
\caption{Gender Distribution by Year}
\label{Gender}
\begin{tabular}{l|l|l}
Year & Men & Women \\ \hline
2013 & 49  & 14    \\ \hline
2014 & 44  & 12    \\ \hline
2015 & 58  & 16    \\ \hline
2016 & 60  & 19    \\ \hline
2017 & 65  & 26    \\ \hline
2018 & 90  & 36   
\end{tabular}
\end{table}

\begin{figure}[H]
\centering
\includegraphics[width=\textwidth]{gender.png}
\caption{Graph}
\label{fig:Gender}
\end{figure}

\begin{table}[H]
\centering
\caption{Distribution of Ethnicity by Year}
\label{Ethnicity}
\begin{tabular}{c|c|c|c|c|c|c|c}
Year & Caucasian & Oriental & Hispanic & African American & Native American & Multiple Selection & Undeclared \\ \hline
2013 & 43        & 7		& 7        & 3                & 1               & 2                  & 0          \\ \hline
2014 & 43        & 5		& 4        & 2                & 1               & 1                  & 0          \\ \hline
2015 & 47        & 9        & 10       & 4                & 1               & 1                  & 2          \\ \hline
2016 & 53        & 9		& 9        & 1                & 7               & 0                  & 0          \\ \hline
2017 & 60        & 12		& 3        & 5                & 5               & 6                  & 0          \\ \hline
2018 & 91        & 8        & 12       & 3                & 4               & 8                  & 0         
\end{tabular}
\end{table}

\begin{figure}[H]
\centering
\includegraphics[width=\textwidth]{ethnicity.png}
\caption{Graph}
\label{fig:Ethnicity}
\end{figure}


\section*{Discussion}

I think gender and racial diversity are both very important in general, as well as in regard to the Computer Science and 
Engineering department here at Notre Dame. Gender and ethnic diversity give us the opportunity to draw from as many 
different points of view as possible, which is very important in helping solve problems in the everyday lives of different 
people. I think that while there is no easy way to increase diversity, the department here at Notre Dame, as well 
as computer science programs around the country should work to fight the stigma that computer science is exclusively 
for white males, and that anyone can be a programmer. The numbers paint a very vivid picture that, while more women 
begining to take an interest in computer science here(though at a slower rate then men), the same cannot be said for minorities. This is exactly why I think 
they should help fight the common computer science stereotypes because it makes the major seem less appealing to 
non-white, non-male students.

I think the computer science and engineering department here does a very good job of being welcoming to all kinds 
of people and I'm not sure there is much they could be doing here to improve the situation. I think the problem begins 
before students ever step foot in a departmental class, but when they decide their majors or intended majors either 
senior year of high school or freshman year here. 

Overall I would say that while diversity (both of gender and ethnicity) is severely lacking in the computer science and 
engineering department here at Notre Dame, I believe there is no easy way to increase it by our own actions here on campus, but it 
is a nationwide problem that begins with changing the public perception of what it means to be a computer science student 
and who is able to be one (the answer is everyone).



\end{document}